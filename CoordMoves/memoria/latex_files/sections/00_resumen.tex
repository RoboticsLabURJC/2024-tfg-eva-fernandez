\chapter*{Resumen}

El mundo de la robótica está en auge, y es por eso que cada vez son más visibles robots de todo tipo: Aspiradoras, friegasuelos, AMR's, cobots, robots que ayudan en cirugías, etc.

Sin embargo, y aunque también están en auge, los robots humanoides no están tan evolucionados como sus compañeros con ruedas. Esto es por la dificultad que tiene construir y programar estos robots, ya que, no sólo deben ser estables tanto dinámica como estáticamente, si no que hay que coordinar todas sus articulaciones para hacerles capaces de cumplir diferentes tareas, ya sea andar, subir escaleras, abrir una puerta, servir un vaso de agua, etc.

El objetivo de este TFG es mostrar la complicación que esto conlleva y ser capaces de construir una librería fácil para que el usuario sea capaz de programar al robot humanoide NAO de manera sencilla y convertirlo en un robot de servicio para diferentes aplicaciones, utilizando las funciones y clases de dicha librería. Esto se hará también para demostrar la potencia de la librería.

Este proyecto se ha desarrollado de manera completamente simulada, utilizando entornos virtuales para el diseño, implementación y validación del sistema, sin necesidad de hardware físico durante el proceso. Esto además nos da un abanico aún más amplio de opciones a la hora de diseñar el escenario para la aplicación de servicios que ofrecerá NAO, en este caso un invernadero.


\vspace{.5cm}

\textbf{Palabras clave:} robótica, coordinación, articulaciones, robot de servicios, ROS 2, librería, abstracción, usuario, TFG, NAO. 